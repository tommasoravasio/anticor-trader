\documentclass[11pt]{article}
\usepackage[margin=0.75in]{geometry}
\usepackage{amsmath}
\usepackage{booktabs}
\usepackage{graphicx}
\usepackage{hyperref}
\usepackage{array}
\usepackage{siunitx}
\usepackage{multirow}

\title{\vspace{-1cm}ANTICOR Algorithm: Empirical Validation and Performance Analysis}
\author{\href{https://github.com/tommasoravasio/anticor-trader}{Github Link to the project}}
\date{}

\begin{document}
\maketitle
\vspace{-0.5cm}

\noindent
This report presents empirical results from backtesting the ANTICOR algorithm (Borodin et al., 2004) and its enhanced variants ANTI$^1$ (smoothed) and ANTI$^2$ (composed) across multiple market scenarios. The ANTICOR algorithm exploits negative correlations between assets to construct portfolios that transfer wealth from strong-performing to weak-performing assets, aiming to capture mean-reversion patterns. Using two consecutive time windows, it computes a claim matrix based on cross-correlations and relative performance, then rebalances the portfolio accordingly.

\noindent Our analysis is conducted under several simplifying assumptions. In particular, performance is sensitive to the choice of the window parameter $W$, transaction costs are not modeled, and daily portfolio rebalancing is assumed to be feasible. These limitations should be kept in mind when interpreting the empirical results.

\section{Experimental Design}

We validated the algorithm across 7 diverse datasets to assess robustness under different market conditions:

\begin{table}[h]
\centering
\small
\begin{tabular}{@{}lp{8cm}@{}}
\toprule
\textbf{Dataset} & \textbf{Description} \\
\midrule
SP500 Top25 & 25 largest S\&P 500 stocks by market cap \\
SP500 25 Diversified & 25 sector-diversified S\&P 500 stocks \\
NYSE Top25 & 25 largest NYSE-listed stocks \\
25MA & 25  multi-asset diversified portfolio \\
25 ETFs & Diversified ETF portfolio \\
25 Maximum Factors & High-volatility, speculative assets \\
15 Risk Diversified & Mixed risk profile portfolio \\
\midrule
Trend Dominant & Synthetic: one clear trending winner \\
Mean Reversion & Synthetic: assets oscillate with frequent leadership flips \\
Rotating Leaders & Synthetic: winner rotates every $\sim$60 days \\
Choppy Blocks & Synthetic: block-level drifts with rank reshuffling \\
\bottomrule
\end{tabular}
\end{table}

\noindent
The backtesting period spans approximately 5 years (August 2019 - January 2025), covering 1,276 trading days. Data sourced from CRSP via WRDS includes proper handling of delisting returns. Window parameter $W=30$ is used consistently. Performance is benchmarked against Equal-Weighted Portfolio (EWP). Additionally, 4 synthetic scenarios (25 assets, 600 days) are used for out-of-sample validation. 

\section{Results Analysis}


\begin{table}[h]
\centering
\small
\begin{tabular}{@{}lcccccc@{}}
\toprule
\textbf{Dataset} & \textbf{Strategy} & \textbf{Cum. Ret.} & \textbf{Ann. Ret.} & \textbf{Sharpe} & \textbf{Max DD} & \textbf{Calmar} \\
\midrule
\multirow{4}{*}{\textbf{25 ETFs}} 
& ANTICOR & 45\% & 8.0\% & 0.24 & -70.1\% & 0.11 \\
& ANTI$^1$ & 132\% & 19.1\% & 0.75 & -56.4\% & 0.34 \\
& \textbf{ANTI$^2$} & \textbf{532\%} & \textbf{49.4\%} & \textbf{2.42} & \textbf{-11.0\%} & \textbf{4.51} \\
& EWP & 48\% & 8.5\% & 0.69 & -20.4\% & 0.42 \\
\midrule
\multirow{4}{*}{\textbf{NYSE Top25}} 
& ANTICOR & 172\% & 30.8\% & 1.11 & -27.2\% & 1.13 \\
& ANTI$^1$ & 160\% & 29.3\% & 1.40 & -15.1\% & 1.94 \\
& \textbf{ANTI$^2$} & \textbf{199\%} & \textbf{36.8\%} & \textbf{1.51} & \textbf{-18.4\%} & \textbf{2.00} \\
& EWP & 85\% & 17.9\% & 1.29 & -14.9\% & 1.20 \\
\midrule
\multirow{4}{*}{\textbf{SP500 Top25}} 
& ANTICOR & 220\% & 27.2\% & 0.60 & -58.1\% & 0.47 \\
& ANTI$^1$ & 370\% & 37.8\% & 1.06 & -44.4\% & 0.85 \\
& \textbf{ANTI$^2$} & \textbf{434\%} & \textbf{44.1\%} & \textbf{1.19} & \textbf{-47.3\%} & \textbf{0.93} \\
& EWP & 253\% & 29.9\% & 1.34 & -25.9\% & 1.15 \\
\midrule
\multirow{4}{*}{\textbf{25MA}} 
& ANTICOR & -41\% & -10.2\% & -0.15 & -88.7\% & -0.12 \\
& ANTI$^1$ & 47\% & 8.4\% & 0.16 & -80.9\% & 0.10 \\
& ANTI$^2$ & 152\% & 22.3\% & 0.38 & -85.0\% & 0.26 \\
& \textbf{EWP} & \textbf{163\%} & \textbf{22.2\%} & \textbf{1.03} & \textbf{-26.3\%} & \textbf{0.84} \\
\bottomrule
\end{tabular}
\caption{Performance comparison across selected datasets. Bold indicates best strategy for each dataset.}
\label{tab:results}
\end{table}


The 25 ETFs portfolio achieves the best results (ANTI$^2$: 532\% return, Sharpe 2.42, Calmar 4.51), benefiting from decorrelated asset classes. NYSE Top25 and SP500 Top25 show strong risk-adjusted gains with Sharpe $>1.0$. However, momentum-driven datasets (25MA, Maximum Factors) prove challenging---here EWP often outperforms, revealing ANTICOR's vulnerability to persistent trends and high-volatility regimes.

\section{Synthetic Data Validation}

Following our analysis, we hypothesized that the algorithm beats the best stock only in markets with alternating winners. To test this, we created four synthetic datasets with features designed to enforce different market behaviors.
Table \ref{tab:synthetic} presents annualized returns across 4 controlled synthetic scenarios.

\begin{table}[h]
\centering
\small
\begin{tabular}{@{}lcccc@{}}
\toprule
\textbf{Scenario} & \textbf{ANTICOR} & \textbf{ANTI$^1$} & \textbf{ANTI$^2$} & \textbf{Benchmark} \\
\midrule
Trend Dominant & 0.197 & 0.296 & 0.265 & \textbf{0.443} \\
Mean Reversion & 0.202 & \textbf{0.540} & 0.508 & -0.026 \\
Rotating Leaders & -0.407 & -0.282 & -0.229 & \textbf{-0.192} \\
Choppy Blocks & -0.238 & -0.129 & -0.103 & \textbf{-0.039} \\ \\
\bottomrule
\end{tabular}
\caption{Annualized returns across synthetic scenarios. Bold indicates best performer.}
\label{tab:synthetic}
\end{table}

\noindent
In \textbf{mean-reversion} environments, ANTICOR strongly outperforms the benchmark, confirming that reversal dynamics create ideal conditions for the strategy. In contrast, in \textbf{trend-dominant} markets, Buy \& Hold achieves the highest Sharpe ratio, while all ANTICOR variants underperform, as expected in the presence of a persistent leader. In the \textbf{rotating leaders} scenario, ANTICOR does not outperform Buy \& Hold, suggesting that frequent leadership changes alone are not sufficient to generate excess returns. This result is reinforced in the \textbf{choppy blocks} setting, where despite high turnover and frequent rank reshuffling, Buy \& Hold continues to outperform the strategy. For simplicity, we deem the strategy successful if any variant (ANTICOR, ANTI$^1$, or ANTI$^2$) outperforms Buy \& Hold; a deeper variant-level analysis is left for future work.

\section{Conclusions and Limitations}

ANTICOR and its variants demonstrate significant alpha in mean-reverting, decorrelated markets (ETFs, diversified indices) but underperform in momentum-driven or persistently trending regimes. The recursive ANTI$^2$ composition consistently improves upon vanilla ANTICOR, suggesting that meta-learning across window experts captures additional structure. 
Overall, these results indicate that ANTICOR's outperformance is driven by genuine mean-reversion rather than by high turnover or alternating leadership alone. When such reversal dynamics are absent, Buy \& Hold remains the dominant strategy.
\noindent Future work should further investigate the proposed hypothesis, explore regime-detection methods for dynamic strategy switching, and relax simplifying assumptions, including the explicit modeling of transaction costs.

\end{document}
